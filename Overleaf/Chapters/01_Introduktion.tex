\documentclass[../main.tex]{subfiles}
\usepackage{tabularx}
\usepackage{graphicx}
\usepackage{array}

\begin{document}

\chapter{Introduktion}
Projektet i kurset \emph{62768 - Elektriske Energisystemer Projekt}, består grundlæggende af at kunne lave et stabilt og effektivt forsyningssystem, der kan levere den krævede spænding hele tiden, og skifte mellem to forskellige forsyningskilder (solceller og motor).  Projektet indebærer en basal forståelse for elektriske energisystemer, der omfatter alt fra produktion, fordeling og styring af elektrisk energi. 

Projektet er meget relevant i lyset af den nuværende klimakrise og de stigende energipriser, og vil fortsat være i fremtiden. Det er vigtigt at kunne levere energi på en stabil måde, fra mange forskellige energikilder spredt ud over hele verden, og med høj effektivitet. Derudover, med fremkomsten af elektriske køretøjer, smarte hjem og andre teknologier, der afhænger af elektrisk energi, er det afgørende at have en dybtgående forståelse af elektriske systemer og deres styring.

Denne rapport beskriver vores analyse, design, implementation og test af projektet.

\section{Problemformulering}
Der skal laves et system, som kan regulere den rette spænding fra solceller, og når der ikke kan leveres nok energi igennem solcellerne, skal en generator overtage spændingsforsyningen, således der ikke kommer uregelmæssigheder i de tilsluttede apparater. 

Der skal derfor også overvejes hvordan systemet skal bygges op og hvordan, vi vil bruge de lærte teknikker og metoder fra tidligere kurser, til at sørge for at det bliver et stabilt og brugbart system. Der skal også vurderes yderligere ting som kan være relavant for projektet, såsom hvordan energien skal lagres, og hvordan der fås mest mulig brugbar energi fra solcellerne. Kravspecifikationerne for projektet er vedlagt under reference punktet.

\section{Kravspecifikation}
Projektet har følgende krav som er rangeret efter nødvendighed.

\renewcommand{\arraystretch}{1.3}
 % \resizebox{\textwidth}{!}{%
    \begin{tabular}{|l|l|p{0.4\textwidth}|l|}
 \hline
 \rowcolors{HTML}{c0c0c0c0}\textbf{Nr.} & \textbf{Navn} & \textbf{Beskrivelse} &\textbf{Prioritet}\\
 \hline
  1  & AC-generator spænding (V1) & Den ensrettede spænding på lade-
kondensatoren (V1) skal være 15
volt +/- 1 volt. For en buck-strøm på
100 mA & 1 \\
 \hline
 2  & item  & item & item \\
 \hline
 3  & item  & item & item \\
 \hline
 4  & item  & item & item \\
 \hline
 5  & item  & item & item \\
 \hline
 6  & item  & item & item \\
 \hline
 7  & item  & item & item \\
 \hline
 8  & item  & item & item \\
 \hline
 9  & item  & item & item \\
 \hline
 10  & item  & item & item \\
 \hline
 11  & item  & item & item \\
 \hline
 12  & item  & item & item \\
 \hline
 12  & item  & item & item \\
 \hline
 13  & item  & item & item \\
 \hline
 14  & item  & item & item \\
 \hline
 15  & item  & item & item \\
 \hline
 16  & item  & item & item \\
 \hline
 17  & item  & item & item \\
 \hline
 18  & item  & item & item \\
 \hline
 19  & item  & item & item \\
\hline
    \end{tabular}

\end{document}