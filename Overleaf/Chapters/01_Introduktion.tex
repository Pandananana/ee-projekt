\documentclass[../main.tex]{subfiles}
\usepackage{tabularx}
\usepackage{graphicx}
\usepackage{array}

\begin{document}

\chapter{Introduktion}
Projektet i kurset \emph{62768 - Elektriske Energisystemer Projekt}, består grundlæggende af at kunne lave et stabilt og effektivt forsyningssystem, der kan levere den krævede spænding hele tiden, og skifte mellem to forskellige forsyningskilder (solceller og motor).  Projektet indebærer en basal forståelse for elektriske energisystemer, der omfatter alt fra produktion, fordeling og styring af elektrisk energi. 

Projektet er meget relevant i lyset af den nuværende klimakrise og de stigende energipriser, og vil fortsat være i fremtiden. Det er vigtigt at kunne levere energi på en stabil måde, fra mange forskellige energikilder spredt ud over hele verden, og med høj effektivitet. Derudover, med fremkomsten af elektriske køretøjer, smarte hjem og andre teknologier, der afhænger af elektrisk energi, er det afgørende at have en dybtgående forståelse af elektriske systemer og deres styring.

Denne rapport beskriver vores analyse, design, implementation og test af projektet.

\section{Problemformulering}
Der skal laves et system, som kan regulere den rette spænding fra solceller, og når der ikke kan leveres nok energi igennem solcellerne, skal en generator overtage spændingsforsyningen, således der ikke kommer uregelmæssigheder i de tilsluttede apparater. 

Der skal derfor også overvejes hvordan systemet skal bygges op og hvordan, vi vil bruge de lærte teknikker og metoder fra tidligere kurser, til at sørge for at det bliver et stabilt og brugbart system. Der skal også vurderes yderligere ting som kan være relavant for projektet, såsom hvordan energien skal lagres, og hvordan der fås mest mulig brugbar energi fra solcellerne. Kravspecifikationerne for projektet er vedlagt under reference punktet.
\newpage
\section{Kravspecifikation}\label{Sec:Kravspecifikation}
Projektet har følgende krav.
\renewcommand{\arraystretch}{1.2}
 % \resizebox{\textwidth}{!}{%
    \begin{longtable}{|l|p{0.34\textwidth}|p{0.4\textwidth}|l|}
 \hline
 \rowcolors{HTML}{c0c0c0c0}\textbf{Nr.} & \textbf{Navn} & \textbf{Beskrivelse} &\textbf{Prioritet}\\
 \hline
  1  & AC-generator spænding (V1) & Den ensrettede spænding på lade-
kondensatoren (V1) skal være 15
volt +/- 1 volt. For en buck-strøm på
100 mA & 1 \\
 \hline
 2  & AC-generator strøm  & AC-generator systemet skal kunne
levere 300 mA til buck konverteren. & 1 \\
 \hline
 3  & AC-generator spænding (V1) under
belastning  & Ved en belastningsændring fra 100
mA til 300 mA skal spændingen (V1)
overholde krav 1, indenfor 1.0 s. & 1 \\
 \hline
 4  & Belastningsspændingen (V2)   & Spændingen V2 skal være indenfor
+/- 1.0 volt & 1 \\
 \hline
 5  & Belastningsstrøm (V2)   & Spændingskilde V2 skal kunne levere
levere min. 150 mA. & 1 \\
 \hline
 6  & Belastningsspændingen (V2) under
belastning  & Spændingen V2 skal være indenfor
+/- 1.0 volt ved en
belastningsændring fra 50 mA til 150
mA indenfor 1.0 s. & 1 \\
 \hline
 7  & Energi-lager strøm (V3)  & Spændingen V3 skal være indenfor 5
v +/- 0.5 volt ved strømme større
end 20 mA . & 1 \\
 \hline
 8  & Spændingskonvertering   & Alle spændingskonverteringer skal
udføres med egen konstrueret
buck/boost konverters. & 1 \\
 \hline
 9  & Spændingskonvertering   & Buck/boost konverters og MPPT-
systemet skal implementeres med
diskrete komponenter. & 1 \\
 \hline
 10  & Strømmåling   & Strømmåling kredsløb skal
implementeres med diskrete
komponenter – ingen integrere
kredsløb tilladt (med undtagelse af
op-amps). & 1 \\
 \hline
 11  & Energikilde prioritering  & Under belastning skal energien fra
solcelle-panelet forbruges først. AC-
generatoren fungerer således som
supplerende energikilde. & 1 \\
 \hline
 12  & Boost og Buck konverter   & Skal realiseres som en del af
projektet. & 1 \\
 \hline
 13  & Systemovervågning  & Et PC-baseret
spændingsovervågningssystem
udvikles. & 2 \\
 \hline
 14  & Systemovervågning   & Spændingsovervågningssystemet
opdateres hver 1.0 s. & 2 \\
 \hline
 15  & Muligt software design   & Brug Arduinoens ADC, med timer
interrupts og PWM & 2 \\
 \hline
 16  & Systemtest   & Systemet testes med 100w led light
(0.5 meter) fra solpanelet & 1 \\
 \hline
 17  & Lineær regulator   & Må udskiftes med en buck regulator & 2 \\
 \hline
 18  & Data behandlingsplatform   & Arduino & 2 \\
 \hline
 19  & Komponenttyper generelt  & Generelt skal alle kredsløb implementeres med diskrete komponenter – ingen integrere kredsløb tilladt (med undtagelse af op-amps). Ligeledes må der ikke anvendes færdige sensorer, ud over modstande koblet som en sensor. & 1 \\
\hline
\end{longtable}
\end{document}