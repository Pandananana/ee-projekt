\documentclass[../main.tex]{subfiles}
\begin{document}
\chapter{Konklusion} \label{Chap:Konklusion}
Gennem dette projekt har vi udviklet et robust og effektivt forsyningssystem, der kan levere konstant spænding og skifte mellem to et solcelle-panel og en AC generator. Vi har anvendt vores viden om elektriske energisystemer og reguleringsteknik til at designe, implementere og teste et system, der kan håndtere skiftende energikilder uden at forstyrre de tilsluttede apparater.

Som en del af projektet blev der stillet en kravspecifikation med 19 punkter, hvoraf 17 punkter er opfyldt. Der blev vist i Kapitel \ref{Chap:Kraftværk} at krav 1 til 3 er overholdt, i Kapitel \ref{Chap:Forbruger} at krav 4 til 7 er overholdt, i Kapitel \ref{Chap:Energilagring} at krav 7 er overholdt og i Kapitel \ref{Chap:Systemintegration} at Krav 11 og 16 er overholdt. Krav 8, 9, 12 og 19 overholdes da alle spændingskonverteringer bliver udført med egen konstrueret buck/boost konvertere og MPPT system der er implementeret med diskrete komponenter. Derudover opfyldes Krav 10 da strømmålingskredsløbet er implementeret med diskrete komponenter og dermed ingen integreringskredsløb (bortset fra Op-amps). Krav 15 og 18 overholdes da data behandlingsplatform er Arduino med ADC, Timer og PWM. Krav 17 er overholdt da vi udskiftede den lineær regulator med en buck konverter. De eneste krav der ikke overholdes er dem der har med systemovervågning at gøre. 

I lyset af den nuværende klimakrise og stigende energipriser, samt den voksende afhængighed af elektrisk energi i teknologier som elektriske køretøjer og smarte hjem, er vores projekt meget relevant. Vi har opnået en dybere forståelse af elektriske systemer og deres styring, hvilket vil være afgørende for fremtidige teknologiske fremskridt og bæredygtige energiløsninger.
\end{document}