\documentclass[../main.tex]{subfiles}
\begin{document}

\chapter{Solcellesystem} \label{Chap:Solcellesystem}

\section{Batteri}

    \subsection{Designovervejelser}

\section{Maximum Power Point Tracking}

    \subsection{Introduktion}

For at sikre kontinuerlig maksimal effekt fra solcellen anvendes et MPPT-system til at overvåge for varierende forhold og optimere systemet i realtid. Arbejdspunktet for maksimal effekt findes i en graf, der viser forholdet mellem effekt og spænding, som bruges til at korrigere for den nødvendige spænding. Systemet er blevet realiseret med en Arduino, der overvåger udgangen fra solcellen og styre en buck-konverter til at justere spændingen.

    \subsection{MPPT algoritme}
    
    For at implementere en algoritme der kan overvåge MPP-punktet, benyttes en Arduino ATMEGA2560. Arduionen modtager ADC-målinger fra solcellen og tager udgangspunkt i en modificeret Pertubation & Observation algoritme. 


    \subsection{MPPT buck}
        Dette delkredsløb kontrolleres af MPPT-algoritmen og anvendes til at regulere indgangsstrøm/spændingen fra solcellen så effekten kan tilpasses og optimeres efter det påsatte load.
        
        \subsubsection{Delkrav}

            \begin{enumerate}
                \item Skal kunne regulere forholdet mellem sin egen indgangs- og udgangsspædning samt indgangs-/udgangsstrøm som en funktion af et PWM-kontrolsignal.
                \item Skal kunne håndtere en indgangspændning på op til 23V
                \item Skal kunne håndtere en indgangsstrøm på op til 0.66A
                \item Må ikke trække mere end 20 mA fra PWM-signalet.
                \item Skal være effektiv nok til at tillade videre regulering af udgangsspændingen til 5V og herefter kunne oplade en kondensator.
            \end{enumerate}
            
        
        \subsubsection{Designovervejelser}
        
        \subsubsection{Implementering}
        
        \subsubsection{Test}
        
        \subsubsection{delkonklusion}

    \subsection{MPPT strøm/spændingsmåler}

        \subsubsection{Delkrav}
        
        \subsubsection{Designovervejelser}
        
        \subsubsection{Implementering}
        
        \subsubsection{Test}
        
        \subsubsection{delkonklusion}
    
        
        
\section{Udgangsconverter}

    \subsection{Delkrav}
    
    \subsection{Designovervejelser}
    
    \subsection{Implementering}
    
    \subsection{Test}
    
    \subsection{Delkonklusion}


\end{document}