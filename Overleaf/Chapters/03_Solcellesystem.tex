\documentclass[../main.tex]{subfiles}
\begin{document}

\chapter{Solcellesystem} \label{Chap:Solcellesystem}

\section{Batteri}

    \subsection{Designovervejelser}

\section{Maximum Power Point Tracking}

    \subsection{Introduktion}

For at sikre kontinuerlig maksimal effekt fra solcellen anvendes et MPPT-system til at overvåge for varierende forhold og optimere systemet i realtid. Et MPPT-system gør det muligt hele tiden at opnå den størst mulige effekt fra solcellen, således at så lidt effekt som muligt går til spilde. Systemet er blevet realiseret med en Arduino, der overvåger udgangen fra solcellen og genererer et PWM-signal til en buck-konverter som derved kan justere spænding og strøm for at opnå det optimale forhold. 

    \subsection{MPPT algoritme}

    For at opnå maksimal energiudbytte fra solcellen, benyttes en MPPT-algoritme som er blevet implementeret på en Arduino ATMEGA2560. Algoritmen har til formål at overvåge solcelle-systemet ved at måle effekten fra solcellen, finde det maksimale effekt punkt og levere et passende PWM signal til en buck-konverter. Der er mange forskellige typer algoritmer til at optimere energiudbytte fra en solcelle, og i dette projekt er der taget udgangspunkt i en P\&O (Pertub \& Observe) algoritme, som er blevet modificeret en smule. 
    
    Arduionen modtager ADC-målinger fra solcellen, som består i en strømmåling og en spændingsmåling fra et analogt kredsløb. Ud fra dette beregner programmet effekten, husker den forrige effektværdi, og finder forskellen mellem dem. På denne måde er det muligt hele tiden at spore hvilken vej effekten bevæger sig, og korrigere duty-cycle på PWM-signalet til at ligge omkring et arbejdspunkt for maksimal effekt, som bruges af buck-konverteren. Algoritmen afviger fra den originale P\&O-algoritme ved ikke direkte at se på forskelle i spændingsmålinger, og derudover have en variabel til at skifte retning på hvor PWM-signalet bevæger sig hen afhængig af dens placering på den karakteristiske effekt graf for solcellen (vis graf?) Et flowchart for algoritmen er vist i 'flowchart billede...'. 
    PWM-signalets brug er forklaret i afsnittet \textit{MPPT buck}.

    Undervejs er der blevet lavet mange ændringer i algoritmen og opsætningen af de forskellige elementer der bruges i programmet. Vi er stødt på mange udfordringer mht. opsætning og behandling af analoge måleværdier, timere generel stabilitet af systemet, og har måtte bruge en del tid på at overkomme disse problemer. Der er til sidst i projektet fundet en effektiv løsning som giver os den ønsket effekt af et MPPT-system.


    \subsection{MPPT buck}
        Dette delkredsløb kontrolleres af MPPT-algoritmen og anvendes til at regulere indgangsstrøm/spændingen fra solcellen så effekten kan tilpasses og optimeres efter det påsatte load.
        
        \subsubsection{Delkrav}

            \begin{enumerate}
                \item Skal kunne regulere forholdet mellem sin egen indgangs- og udgangsspædning samt indgangs-/udgangsstrøm som en funktion af et PWM-kontrolsignal.
                \item Skal kunne håndtere en indgangspændning på op til 23V
                \item Skal kunne håndtere en indgangsstrøm på op til 0.66A
                \item Må ikke trække mere end 20 mA fra PWM-signalet.
                \item Skal være effektiv nok til at tillade videre regulering af udgangsspændingen til 5V og herefter kunne oplade en kondensator.
            \end{enumerate}
            
        \subsubsection{Designovervejelser}
            
            Switch
            I valg af switch overvejes forskellige opsætninger af switches som vist 
            
            
            Spole
            
            
            
            Kondensator
            
            
            
            Diode
            
            
            
            
        \subsubsection{Implementering}
            
            
            
            
        \subsubsection{Test}
            
            
            
            
        \subsubsection{delkonklusion}
            
            
            
            
    \subsection{MPPT strøm/spændingsmåler}
            
        \subsubsection{Delkrav}
            
        \subsubsection{Designovervejelser}
            
        \subsubsection{Implementering}
            
        \subsubsection{Test}
            
        \subsubsection{delkonklusion}
            
            
                    
\section{Udgangsconverter}
        
    \subsection{Delkrav}
        
    \subsection{Designovervejelser}
            
            Switch

            Spole

            Kondensator

            Diode


            
    \subsection{Implementering}
        
    \subsection{Test}
        
    \subsection{Delkonklusion}
        
        
\end{document}