\documentclass[../main.tex]{subfiles}
\begin{document}

\chapter{Solcellesystem} \label{Chap:Solcellesystem}

\section{Batteri}

    \subsection{Designovervejelser}

\section{Maximum Power Point Tracking}

    \subsection{Introduktion}

    For at sikre kontinuerlig maksimal effekt fra solcellen anvendes et MPPT-system til at overvåge for varierende forhold og optimere systemet i realtid. Arbejdspunktet for maksimal effekt findes i en graf, der viser forholdet mellem effekt og spænding, som bruges til at korrigere for den nødvendige spænding. Systemet er blevet realiseret med en Arduino, der overvåger udgangen fra solcellen og styre en buck-konverter til at justere spændingen.

    \subsection{MPPT algoritme}
    
    Koden der er blevet implementeret på en Arduino ATMEGA2560, tager ADC tager højde for ændringer i solcellens omgivelser og lysindfald. 


    \subsection{MPPT buck}
    
        \subsubsection{Delkrav}
        
        \subsubsection{Designovervejelser}
        
        \subsubsection{Implementering}
        
        \subsubsection{Test}
        
        \subsubsection{delkonklusion}
        
        
\section{Udgangsconverter}

    \subsection{Delkrav}
    
    \subsection{Designovervejelser}
    
    \subsection{Implementering}
    
    \subsection{Test}
    
    \subsection{Delkonklusion}


\end{document}